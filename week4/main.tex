    \documentclass[answers]{exam}
\newif\ifanswers
\answerstrue% comment out to hide answers

\usepackage{lastpage} % Required to determine the last page for the footer
\usepackage{extramarks} % Required for headers and footers
\usepackage[usenames,dvipsnames]{color} % Required for custom colors
\usepackage{graphicx} % Required to insert images
\usepackage{listings} % Required for insertion of code
\usepackage{courier} % Required for the courier font
\usepackage{lipsum} % Used for inserting dummy 'Lorem ipsum' text into the template
\usepackage{enumerate}
\usepackage{subfigure}
\usepackage{booktabs}
\usepackage{amsmath, amsthm, amssymb}
\usepackage{hyperref}
\usepackage{datetime}
\usepackage{minted}
\settimeformat{ampmtime}
\usepackage{caption}
\usepackage[default]{fontsetup}
\usepackage[english, cherokee]{babel}

\usepackage{tikz}
\usetikzlibrary{positioning,patterns,fit,calc}
% Margins
\topmargin=-0.45in
\evensidemargin=0in
\oddsidemargin=0in
\textwidth=6.5in
\textheight=9.0in
\headsep=0.25in

\linespread{1.1} % Line spacing

% Set up the header and footer
%\pagestyle{fancy}
%\rhead{\hmwkAuthorName} % Top left header
%\lhead{\hmwkClass: \hmwkTitle} % Top center head
%\lfoot{\lastxmark} % Bottom left footer
%\cfoot{} % Bottom center footer
%\rfoot{Page\ \thepage\ of\ \protect\pageref{LastPage}} % Bottom right footer
%\renewcommand\headrulewidth{0.4pt} % Size of the header rule
%\renewcommand\footrulewidth{0.4pt} % Size of the footer rule

\pagestyle{headandfoot}
\runningheadrule{}
\firstpageheader{CS 224n}{Assignment 4}{}
\runningheader{CS 224n} {Assignment 4} {Page \thepage\ of \numpages}
\firstpagefooter{}{}{} \runningfooter{}{}{}

\setlength\parindent{0pt} % Removes all indentation from paragraphs

%----------------------------------------------------------------------------------------
%	CODE INCLUSION CONFIGURATION
%----------------------------------------------------------------------------------------

\definecolor{MyDarkGreen}{rgb}{0.0,0.4,0.0} % This is the color used for comments
\lstloadlanguages{Python} % Load Perl syntax for listings, for a list of other languages supported see: ftp://ftp.tex.ac.uk/tex-archive/macros/latex/contrib/listings/listings.pdf
\lstset{language=Python, % Use Perl in this example
        frame=single, % Single frame around code
        basicstyle=\footnotesize\ttfamily, % Use small true type font
        keywordstyle=[1]\color{Blue}\bf, % Perl functions bold and blue
        keywordstyle=[2]\color{Purple}, % Perl function arguments purple
        keywordstyle=[3]\color{Blue}\underbar, % Custom functions underlined and blue
        identifierstyle=, % Nothing special about identifiers
        commentstyle=\usefont{T1}{pcr}{m}{sl}\color{MyDarkGreen}\small, % Comments small dark green courier font
        stringstyle=\color{Purple}, % Strings are purple
        showstringspaces=false, % Don't put marks in string spaces
        tabsize=5, % 5 spaces per tab
        %
        % Put standard Perl functions not included in the default language here
        morekeywords={rand},
        %
        % Put Perl function parameters here
        morekeywords=[2]{on, off, interp},
        %
        % Put user defined functions here
        morekeywords=[3]{test},
       	%
        morecomment=[l][\color{Blue}]{...}, % Line continuation (...) like blue comment
        numbers=left, % Line numbers on left
        firstnumber=1, % Line numbers start with line 1
        numberstyle=\tiny\color{Blue}, % Line numbers are blue and small
        stepnumber=5 % Line numbers go in steps of 5
}

\usepackage{hyperref}
\hypersetup{
    colorlinks=true,
    linkcolor=blue,
    filecolor=magenta,      
    urlcolor=blue,
}
 

%----------------------------------------------------------------------------------------
%	NAME AND CLASS SECTION
%----------------------------------------------------------------------------------------

\newcommand{\hmwkTitle}{Assignment \#4} % Assignment title
\newcommand{\hmwkClass}{CS\ 224n} % Course/class
\newcommand{\ifans}[1]{\ifanswers \color{red} \vspace{5mm} \textbf{Solution: } #1 \color{black} \vspace{5mm} \fi}

\input std-macros
\input macros

%----------------------------------------------------------------------------------------
%	TITLE PAGE
%----------------------------------------------------------------------------------------
\qformat{\Large\bfseries\thequestion{}. \thequestiontitle{} (\thepoints{})\hfill}

\title{
\vspace{-1in}
\textmd{\textbf{\hmwkClass:\ \hmwkTitle} \\ \hmwkAuthorName}\\
}
\author{}
%\date{\textit{\small Updated \today\ at \currenttime}} % Insert date here if you want it to appear below your name
\date{}

\setcounter{section}{0} % one-indexing

\begin{document}

\maketitle


This assignment is split into two sections: \textit{Neural Machine Translation with RNNs} and \textit{Analyzing NMT Systems}. The first is primarily coding and implementation focused, whereas the second entirely consists of written, analysis questions. If you get stuck on the first section, you can always work on the second as the two sections are independent of each other. 
% Note that the NMT system is more complicated than the neural networks we have previously constructed within this class and takes about \textbf{6 hours to train on a GPU}. Thus, we strongly recommend you get started early with this assignment. 
The notation and implementation of the NMT system is a bit tricky, so if you ever get stuck along the way, please come to Office Hours so that the TAs can support you. We also highly recommend reading \href{https://arxiv.org/abs/2010.04791}{Zhang et al (2020)} to better understand the Cherokee-to-English translation task, which served as inspiration for this assignment. \newline

\begin{questions}
    \input q1
    \input q2
\end{questions}

\Large{\textbf{Submission Instructions}}

\normalsize
You shall submit this assignment on GradeScope as two submissions -- one for ``Assignment 4 [coding]" and another for `Assignment 4 [written]":
\begin{enumerate}
    \item Run the \texttt{collect\_submission.sh} script on Azure to produce your \texttt{assignment4.zip} file. You can use \href{http://www.hypexr.org/linux_scp_help.php}{scp} to transfer files between Azure and your local computer. \\
    If you are on Windows, run \texttt{collect\_submission.bat}. You can either double-click on the script file or execute it in command line. 
    \item Upload your \texttt{assignment4.zip} file to GradeScope to ``Assignment 4 [coding]".
    \item Upload your written solutions to GradeScope to ``Assignment 4 [written]". When you submit your assignment, make sure to tag all the pages for each problem according to Gradescope's submission directions. Points will be deducted if the submission is not correctly tagged.
\end{enumerate}


\end{document}
